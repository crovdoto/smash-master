\subsubsection{Moves} \label{ult-metaknight-moves}

\paragraph[Jab]{Rapid Jab}
\begin{center}
	\begin{tabular}{| c  c  c | c  c |}
		\hline
		Startup	& Endlag & Total & BKB & KBG \\
		4	&	27	&	45	&	60	&	140	\\
		\hline
	\end{tabular}
	
	\emph{(Rapid Jab Finish)}
\end{center}

While generally unsafe, Meta Knight's \emph{rapid jab} has notable strength in dealing shield pressure, either by the edge of a platform or by crossing up shields using the lengthy skid animation. If it does hit, it may deal up to $20.1\%$. The move does not fill the typical role of a jab, which would be relegated to his \textit{down tilt}, instead.

\paragraph{Forward Tilt}

\paragraph{Up Tilt}

\paragraph{Down Tilt}
\begin{center}
	\begin{tabular}{| c  c  c | c  c  c |}
		\hline
		Startup	& Endlag & Total & BKB & KBG & Trip\\
		3	&	13	&	18	&	15	&	88	& 0.25\\
		\hline
	\end{tabular}
\end{center}
An excellent poking tool. With a 3-frame startup and a $25\%$ chance to trip, Meta Knight's \emph{down tilt} serves as the perfect jab for the character, letting him jab-lock or set up for a combo-starter such as a grab or \textit{dash attack}.
%
\paragraph{Dash Attack}
\begin{center}
	\begin{tabular}{| c  c  c | c  c  c |}
		\hline
		Startup	& Endlag & Total & BKB & KBG & Angle \\
		7	&	21	&	32	&	65	&	107 & 	60\deg \\
		-	&	-	&	-	&	- 	& 	- 	&	70\deg	\\
		-	&	-	&	-	&	67	&	-	&	80\deg	\\
		\hline
	\end{tabular}
\end{center}
Often revered as one of the best (combo-starting) dash attacks in the game, Meta Knight's \emph{dash attack} is one of, if not the most important, moves in his toolkit. The variable angles it can send at set up for various \textit{up air} strings. The downside of this move is that it scales somewhat fast. It ceases to be an effective combo-starter at early-mid percentages, but does not become a potent KO-threat until much later percentages.

Nonetheless, this move is fundamental to Meta Knight's combo game, is hard to react-DI, and always provides at least \textbf{8 frames} of hit advantage.

\paragraph{Forward Smash}
% TODO: fsmash frame data
Due to its absurd shield safety and low endlag, \textit{f-smash} is one of Meta Knight's best grounded neutral options. A potent KO-threat at mid-high percentages that functions greatly as a conditioning tool. After hitting a shield, Meta Knight may throw out a \textit{d-smash} to cover for certain out-of-shield options (e.g. grabs, shield-drops).

\paragraph{Up Smash}

\paragraph{Down Smash}
% TODO: dsmash frame data
The fastest \textit{d-smash} in the game. An excellent ``get-off-me'' option. The unorthodox hitbox shape makes it difficult to hit at times, particularly if you seek to use it as a finisher to your combos, but its speed makes it reliable in what it can do.

\paragraph{Neutral Air}
\paragraph{Forward Air}
\paragraph{Back Air}
Generally a safe spacing aerial with high knockback scaling. Typical finisher for reverse bridges.
\paragraph{Up Air}
\begin{center}
	\begin{tabular}{| c  c  c | c  c  c | c  c |}
		\hline
		Startup	& Endlag & Total & BKB & KBG & Angle &	Landing Lag	&	Shield Stun	\\
		6	&	20	&	26	&	65	&	128 & 	67\deg	&	9 	&	3	\\
		-	&	-	&	-	&	- 	& 	- 	&	50\deg	&	-	&	-	\\
		\hline
	\end{tabular}
\end{center}
The essential move for Meta Knight's ladder combos, as well  being an important aspect of his bridges and stage-carries. While its hitboxes are lacklustre, and the move does not come out as fast as in Brawl, and is certainly out-shined by other up-aerials in Ultimate. In order for ladders to be true, you need to deal 26 frames of hitstun, so that \textit{up air} strings into \emph{up air}. This threshold is approximately $0.236/w$, where $w$ is the opponent's weight:
\begin{center}
	\begin{tabular}{| c c c c |}
		\hline
		\textbf{Character} & Pichu & Mario & Bowser	\\
		\hline
		$w$		&	62		&	98		&	135		\\
		$\%$	&	17.4	&	22.2	&	27.1	\\
		$\%/w$	&	.281	&	.227	&	.200	\\
		\hline
	\end{tabular}
\end{center}
The usual strategy with \textit{up air ladders} is to keep going until it starts sending into tumble, to then finish the opponent off with \textit{shuttle loop}.

\paragraph{Down Air}
While significantly more modest than its \textit{Brawl} counterpart, Meta Knight's \textit{dair} is one of his most versatile options, it works as
	\begin{enumerate}[label = \roman*.]
		\item an out-of-shield option
		\item a footstool option
		\item a combo-extender
		\item a combo-starter
		\item a neutral approach
		\item aerial spacing
		\item an edgeguarding tool
	\end{enumerate}
\paragraph[Neutral Special]{Mach Tornado}
\paragraph[Side Special]{Drill Rush}
\paragraph[Up Special]{Shuttle Loop}
\paragraph[Down Special]{Dimensional Cape}
\paragraph{Grabs}
As one of his most vital whiff-punishes and approaches alongside \textit{dash attack}, Meta Knight has one of the unfortunately small grab-lengths. At a mere 11.5 units, his grab is tied with the likes of \textit{Wii Fit Trainer, Ike \& Diddy Kong}. Thankfully, his ground speed makes up for this. If Meta Knight does manage to grab an opponent, it could quickly become a touch-of-death, as both \textit{d-throw} and \textit{f-throw} have associated zero-to-death combos.
\paragraph{Forward Throw}
As Meta Knight's fastest throw, he is able to convert incorrect DI into a ladder-confirm at mid-percentages. This throw also has some niche usage at early percentages against heavies or fast-fallers.
\paragraph{Backward Throw}
Alongside \textit{up throw}, this is Meta Knight's most-damaging throw. What differentiates it from up-throw, however, is its combo-potential at early percentages.
\paragraph{Up Throw}
An occasional option when all else has failed. KO-ing opponents early off of platforms above makes Meta Knight a more potent threat on stages like \textit{Yoshi's Story \& Battlefield}.
\paragraph{Down Throw}
Meta Knight's most consistent combo-throw. At early-mid percentages it connects quite reliably with most of his toolkit, since a minimum of \textbf{21 frames} is provided as hit advantage.