\subsubsection{Moves} \label{ult-metaknight-moves}

\paragraph[Jab]{Rapid Jab}
\begin{center}
	\begin{tabular}{| c  c  c | c  c |}
		\hline
		Startup	& Endlag & Total & BKB & KBG \\
		4	&	27	&	45	&	60	&	140	\\
		\hline
	\end{tabular}
	
	\emph{(Rapid Jab Finish)}
\end{center}

While generally unsafe, Meta Knight's \emph{rapid jab} has notable strength in dealing shield pressure, either by the edge of a platform or by crossing up shields using the lengthy skid animation. If it does hit, it may deal up to $20.1\%$. The move does not fill the typical role of a jab, which would be relegated to his \textit{down tilt}, instead.

\paragraph{Forward Tilt}

\begin{center}
	\begin{tabular}{| c  c  c | c  c  c |}
		\hline
		Startup	& Endlag & Total & BKB & KBG & Angle \\
		6	&	28	&	30	&	80	&	107	& 361\deg \\
		\hline
	\end{tabular}
	
	\emph{(startup: ftilt1; \ endlag, total, bkb, kbg, angle: ftilt3)}
\end{center}

While it does not grant any combo potential or shield pressure, \textit{ftilt} serves as a decent approach when \textit{dash grab} and \textit{dash attack} cannot reasonably connect. To fully maximise the reach of \textit{ftilt}, it is best done in conjunction with a skid-cancel (especially of the diagonal variant). This may be referred to as a \textit{sliding ftilt}.

To input all three hits rapidly with ease, do the first hit with either the \textit{attack button} or tapping forward on the C-stick using \textit{tilt attacks}, then proceed to press an \textit{attack} or \textit{grab} button twice.

What makes this move useful in any regard is mixing up the opponent's expectation that Meta Knight will either \textit{dash grab} or \textit{dash attack} them. 

It is also not a terrible KO option at ledge, killing most characters at $130\%$ on legal stages. It is susceptible to being SDI'ed \textbf{down and in}, meaning the final hit could miss, especially if done sliding.

\paragraph{Up Tilt}
A mediocre anti-air with no real \textit{scoop hitbox}, this move is a blemish in Meta Knight's ground game. Despite this, it has a solid purpose as a combo starter for ladders, or kill confirms at around $100\%$; if it hits, then it may lead directly into a KO via \textit{Shuttle Loop}. You do have to react to the opponent's directional influence, but it is generally a true combo for all DIs, though at varying percentages.

\paragraph{Down Tilt}
\begin{center}
	\begin{tabular}{| c  c  c | c  c  c |}
		\hline
		Startup	& Endlag & Total & BKB & KBG & Trip\\
		3	&	14	&	18	&	15	&	88	& 0.25\\
		\hline
	\end{tabular}
\end{center}
An excellent poking tool. With a 3-frame startup and a $25\%$ chance to trip, Meta Knight's \emph{down tilt} serves as the perfect jab for the character, letting him jab-lock or set up for a combo-starter such as a grab or \textit{dash attack}.
%
As a jab-lock, with only $14$ frames of endlag, Meta Knight is provided with \textbf{24 frames} of hit advantage. This is \textit{just} enough for $fsmash$ to not work, but still gives him many punitive options in starters like \textit{dash attack} and \textit{dair}, or finishers like \textit{Shuttle Loop} and $dsmash$.

Since \textit{dtilt} halts all momentum, Meta Knight cannot do \textit{sliding dtilts}. In some aspects, this is nice, as you need not worry about an input - which is easy to mess up, but ultimately this is a negative; it limits the potential reach \textit{dtilt} could have had.

\paragraph{Dash Attack}
\begin{center}
	\begin{tabular}{| c  c  c | c  c  c |}
		\hline
		Startup	& Endlag & Total & BKB & KBG & Angle \\
		7	&	21	&	32	&	65	&	107 & 	60\deg \\
		-	&	-	&	-	&	- 	& 	- 	&	70\deg	\\
		-	&	-	&	-	&	67	&	-	&	80\deg	\\
		\hline
	\end{tabular}
\end{center}
Often revered as one of the best (combo-starting) dash attacks in the game, %TODO: possibly italicise "dash attacks" for consistency purposes
 Meta Knight's \emph{dash attack} is one of, if not the most important, moves in his toolkit. The variable angles it can send at set up for various \textit{up air} strings. The downside of this move is that it scales somewhat fast. It ceases to be an effective combo-starter at early-mid percentages, but does not become a potent KO-threat until much later percentages.

Nonetheless, this move is fundamental to Meta Knight's combo game, is hard to react-DI, and always provides at least \textbf{8 frames} of hit advantage.

\paragraph{Forward Smash}
\begin{center}
	\begin{tabular}{| c | c  c  c | c  c  c | c |}
		\hline
		Damage 	&	Startup	& Endlag & Total & BKB 	& KBG 	& 	Angle 	&	Shield Stun	\\
		16		&	24		&	17	&	41	&	30	&	108 & 	361\deg	&	11 		\\
		-		&	-		&	-	&	-	&	- 	& 	98 	&	-		&	-		\\
		\hline
	\end{tabular}
\end{center}
Due to its absurd shield safety and low endlag, \textit{f-smash} is one of Meta Knight's best grounded neutral options. A potent KO-threat at mid-high percentages that functions greatly as a conditioning tool. After hitting a shield, Meta Knight may throw out a \textit{d-smash} to cover for certain out-of-shield options (e.g. grabs, shield-drops).

It can be held at frame 20, meaning it can - at times - be treated as having 4 frames of start up, which makes it excellent for tech-chasing rolls in, or catching ledge options.

The move is minimally weaker on the length of the sword, but this is generally not an issue, as the damage is still uniform on the entirety of the sword.

It should be mentioned that, since $fsmash$ is active for only one frame, its rebound animation will be shorter than most other attacks if it were to clank. Ergo, it allows for punishes. One of these punishes, that ought to be mentioned is the ``Dark Clash'' as popularised by \textit{@ShredTM}. The technique is simply using \textit{dimensional cape attack} after Meta Knight's $fsmash$ rebound animation (which can be buffered).
%todo: put dark clash in techniques

\paragraph{Up Smash}
\begin{center}
	\begin{tabular}{| c  c  c | c  c |}
		\hline
		Startup	& Endlag & Total & BKB & KBG  \\
		8		&	32	&	49	&	65	&	148	 \\
		\hline
	\end{tabular}
	
	\emph{(startup: usmash1; \ endlag, total, bkb, kbg: usmash3)}
\end{center}
This move makes Meta Knight feel like a low-tier character at times. It has a bad scoop like $utilt$, but with significantly less range. Its main benefit is its 8 frame startup, making $dash\ attack$ or $down\ throw$ at $0\%$ a decent combo, which puts the opponent at a nice $20\%$-ish range. Unfortunately, while the move is pretty strong with a 90\deg\. angle, it barely reaches any platforms - with the exceptions of \textit{Hollow Bastion} and \textit{maybe (Small) Battlefield}. Essentially, it is a useless techchase tool, even though it should be one, intuitively.

As a consequence, Meta Knight's best platform techchase is to either \textit{up tilt} or \textit{up air}; the former grants very little reward, and the latter is often a positional risk if the opponent goes for a \textit{getup attack} in anticipation. $Up\ Smash$ is therefore best used as either a finisher or anti-air, with potential utility against taller characters.

\paragraph{Down Smash}
\begin{center}
	\begin{tabular}{| c | c  c  c | c  c  c |}
		\hline
		Damage 	&	Startup	& Endlag & Total & BKB 	& 	KBG & Angle		\\
		10		&	4		&	33	 &	37	&	50	&	77 & 	35\deg 	\\
		13		&	9		&	28	 &	-	&	- 	& 	78 	&	-		\\
		\hline
	\end{tabular}
\end{center}
The fastest \textit{d-smash} in the game. An excellent ``get-off-me'' option. The unorthodox hitbox shape makes it difficult to hit at times, particularly if you seek to use it as a finisher to your combos, but its speed makes it reliable in what it can do.

The unorthodox angle can sometimes work in Meta Knight's favour, since the move comes out fast. With incorrect DI, it will send at a nearly horizontal angle (with gravity) - though the caveat is that with correct DI it will seldom lead to a KO.

\paragraph{Neutral Air}
\begin{center}
	\begin{tabular}{| c | c  c  c | c  c  c | c |}
		\hline
		Damage &	Startup	& Endlag & Total & BKB & KBG & Angle &	Landing Lag	\\
		10	&	6	&	23	&	43	&	40	&	100 & 	361\deg	&	7 		\\
		7.5	&	-	&	-	&	-	&	- 	& 	- 	&	-		&	-		\\
		\hline
	\end{tabular}
\end{center}
The most prominent neutral tool Meta Knight has is his \textit{neutral air}. Its simultaneous utility as both a combo starter, extender, and finisher makes this one of the most useful moves in his kit. When landed perfectly, it will provide at least \textbf{18 frames} of hit advantage, which opens up for a lot of follow-ups like dash attack and grab - especially with the granted \textit{Sakurai angle}. 

\textbf{This may require a citation.} Since Meta Knight's shorthop fastfall is $22$ frames long, we may also have a first-actionable-frame at frame $23$, which means even a shorthop $nair$ fastfall would provide at least 9 frames of hit advantage.
\paragraph{Forward Air}
\paragraph{Back Air}
Generally a safe spacing aerial with high knockback scaling. Typical finisher for reverse bridges.
\paragraph{Up Air}
\begin{center}
	\begin{tabular}{| c  c  c | c  c  c | c  c |}
		\hline
		Startup	& Endlag & Total & BKB & KBG & Angle &	Landing Lag	&	Shield Stun	\\
		6	&	20	&	26	&	65	&	128 & 	67\deg	&	9 	&	3	\\
		-	&	-	&	-	&	- 	& 	- 	&	50\deg	&	-	&	-	\\
		\hline
	\end{tabular}
\end{center}
The essential move for Meta Knight's ladder combos, as well  being an important aspect of his bridges and stage-carries. While its hitboxes are lacklustre, and the move does not come out as fast as in Brawl, and is certainly out-shined by other up-aerials in Ultimate. In order for ladders to be true, you need to deal 26 frames of hitstun, so that \textit{up air} strings into \emph{up air}. This threshold is approximately $0.236/w$, where $w$ is the opponent's weight:
\begin{center}
	\begin{tabular}{| c c c c |}
		\hline
		\textbf{Character} & Pichu & Mario & Bowser	\\
		\hline
		$w$		&	62		&	98		&	135		\\
		$\%$	&	17.4	&	22.2	&	27.1	\\
		$\%/w$	&	.281	&	.227	&	.200	\\
		\hline
	\end{tabular}
\end{center}
The usual strategy with \textit{up air ladders} is to keep going until it starts sending into tumble, to then finish the opponent off with \textit{shuttle loop}.

\paragraph{Down Air}
\begin{center}
	\begin{tabular}{| c  c  c | c  c  c | c  c |}
		\hline
		Startup	& Endlag & Total & BKB & KBG & Angle &	Landing Lag	&	Shield Stun	\\
		4	&	23	&	27	&	30	&	108 & 	35\deg	&	9 	&	3	\\
		-	&	-	&	-	&	- 	& 	- 	&	50\deg	&	-	&	-	\\
		-	&	-	&	-	&	-	&	-	&	140\deg &	-	&	-	\\
		\hline
	\end{tabular}
	
	\emph{The frame data typically states that the third hitbox has an angle of 50 degrees, but since it hits behind, it is effectively 140\deg, instead.}
\end{center}
While significantly more modest than its \textit{Brawl} counterpart, Meta Knight's \textit{dair} is one of his most versatile options, it works as
	\begin{enumerate}[label = \roman*.]
		\item an out-of-shield option
		\item a footstool option
		\item a combo-extender
		\item a combo-starter
		\item a neutral approach
		\item aerial spacing
		\item an edgeguarding tool
	\end{enumerate}
	Now, since \textit{dair} is slightly faster and weaker than \textit{uair}, connecting to itself has a slightly different chart, this time for \textbf{4 frame} hit advantage:
	\begin{center}
		\begin{tabular}{| c c c c |}
			\hline
			\textbf{Character} & Pichu & Mario & Bowser	\\
			\hline
			$w$		&	62		&	98		&	135		\\
			$\%$	&	21.0	&	27.0	&	33.1	\\
			$\%/w$	&	.339	&	.276	&	.245	\\
			\hline
		\end{tabular}
	\end{center}
	What is particularly neat with down air as a move, is with Meta Knight's frame data means that a perfect (buffered) shorthop $dair$ leads to a landing $dair$:
	\begin{gather*}
		SH = 31,\\ dair = 27,\\ SH - dair = 31 - 27 = 4
	\end{gather*}
	Buffered shorthop $uair \to dair$ is similar, being slightly more lenient as \textit{uair} has a total frame count of $26$.
\paragraph[Neutral Special]{Mach Tornado}
\paragraph[Side Special]{Drill Rush}
The move that makes fighting Meta Knight unfathomably scary without a decent recovery. Any Meta Knight who knows how to angle \textit{Drill Rush} in a way that launches the opponent to the bottom blast-zone \textit{will} use this move for \textit{cheese} to its full extent.
\paragraph[Up Special]{Shuttle Loop}
\paragraph[Down Special]{Dimensional Cape}
One of Meta Knight's most devious moves. Cape lets Meta Knight recover, stall, and get out of disadvantage with ambiguity. \textit{Empty cape} allows for invincible movement, while \textit{cape attack} provides a risky, but somewhat powerful hitbox.

The frame data and specifics of the move is a bit difficult to comprehend, as there are essentially 7 variants of it (empty cape, 3 grounded attacks, 3 aerial attacks)

In terms of reach, Dimensional Cape can achieve:
\begin{enumerate}[label = \roman*.]
	\item Vertical reach of approximately 40 units.
	\item Horizontal reach of approximately 60 units.
	\item Diagonal reach of $||(50, 30)|| = \sqrt{3400} \approx 58$ units (angle $\approx 60\deg$)
\end{enumerate}
One may also utilise the C-stick to either:
\begin{enumerate}[label = \roman*.]
	\item micromovement for \textit{Empty Cape}
	\item choosing the direction for \textit{Cape Attack} [requires citation] % TODO: prove this
\end{enumerate}

Auto-cancelling with Dimensional Cape can be somewhat tricky. Especially as it cannot platform-cancel, like the teleportation moves of Mewtwo and Palutena. 
Some setups include:
\begin{enumerate}[label = \roman*.]
	\item fullhop down special + any horizontal movement
	\item down special + up to land on \textit{Town \& City} side platforms
\end{enumerate}

\paragraph{Grabs}
As one of his most vital whiff-punishes and approaches alongside \textit{dash attack}, Meta Knight has one of the smallest grab-lengths in the game. At a mere 11.5 units, his grab is tied with the likes of \textit{Wii Fit Trainer, Ike \& Diddy Kong}. Thankfully, his ground speed makes up for this. If Meta Knight does manage to grab an opponent, it could quickly become a touch-of-death, as both \textit{d-throw} and \textit{f-throw} have associated zero-to-death combos.
\paragraph{Forward Throw}
As Meta Knight's fastest throw, he is able to convert incorrect DI into a ladder-confirm at mid-percentages. This throw also has some niche usage at early percentages against heavies or fast-fallers.
\paragraph{Backward Throw}
Alongside \textit{up throw}, this is Meta Knight's most-damaging throw. What differentiates it from up-throw, however, is its combo-potential at early percentages.
\paragraph{Up Throw}
An occasional option when all else has failed. KO-ing opponents early off of platforms above makes Meta Knight a more potent threat on stages like \textit{Yoshi's Story \& Battlefield}.
\paragraph{Down Throw}
Meta Knight's most consistent combo-throw. At early-mid percentages it connects quite reliably with most of his toolkit, since a minimum of \textbf{21 frames} is provided as hit advantage.